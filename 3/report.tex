\documentclass[a4paper,12pt]{article}
\usepackage{listings}
\begin{document}
\title{「アルゴリズムとデータ構造入門」\\
第3回課題}
\author{工学部情報学科\\
平成25年入学\\
学籍番号:1029-25-2723\\
森井 崇斗 }
\date{\today}
\maketitle

\lstset{numbers=left,basicstyle=\ttfamily\small,
  commentstyle=\textit, keywordstyle=\bfseries}
  \section{繰返型階上のプログラムの作成と実行}
  \subsection{プログラムの内容}

\lstinputlisting{factorial-iter.scm}

\subsection{出力結果}

\fbox{
    \begin{tabular}{l}
$>$ (factorial-iter 103) \\
9902900716486180407546715254581773349090165822114492483005280554699876665841\\
6222832141441073883538492653516385977292093222882134415149891584000000000000\\
000000000000
    \end{tabular}
}

\subsection{説明}

\begin{enumerate}
    \item 初めに引数として{\tt n}を受け取る。
    \item その{\tt n}を利用して、{\tt fact-iter}を呼び出す。
    \item {\tt fact-iter}の引数は3つあり、{\tt counter}を{\tt max-count ( = n )}に等しくなるまで1ずつ大きくしていく。
    \item その{\tt counter}の値を1つずつ掛けた物を{\tt product}に格納する。
    \item 3と4を繰り返すことで$ 1*2*\cdots(n-1)n $を実現する。
\end{enumerate}

\section{教科書1-2-3〜1-3-1の想定質問とその回答}
\subsection{想定質問}

1.3.1で紹介されている{\tt sum}を利用したプログラムを作成せよ。\\
\\
参考
\lstinputlisting{sum.scm}

\subsection{想定質問への解答}

1〜nまでのそれぞれの階乗の和を計算するプログラムを作成する。\\
愚直に1〜nまでの階乗をそれぞれ求め、それの和を取ることでも求めることが可能であるが、$n!$は${n(n-1)!}$であることを利用して実装する\\
$n!$を求める際に既に$(n-1)!$の解は求まっているはずであるので、この解を利用すれば計算量を減らすことが可能である。\\
まず以下に{\tt sum}をそのまま利用した場合の実装例を示す。

\lstinputlisting{fact-sum.scm}

次に前述の方法で計算量を減らした場合の実装例を示す。

\lstinputlisting{fact-sum-smart.scm}

\end{document}

