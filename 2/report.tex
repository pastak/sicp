\documentclass[a4paper,12pt]{article}
\usepackage{listings}
\usepackage{fancybox}
\begin{document}
\title{「アルゴリズムとデータ構造入門」\\
第2回課題}
\author{工学部情報学科\\
平成25年入学\\
学籍番号:1029-25-2723\\
森井 崇斗 }
\date{\today}
\maketitle

\lstset{numbers=left,basicstyle=\ttfamily\small,
  commentstyle=\textit, keywordstyle=\bfseries}

  \section{階上のプログラムの作成と実行}
  \subsection{プログラムの内容}

\lstinputlisting{factrial.scm}

\subsection{出力結果}

\fbox{
    \begin{tabular}{l}
$>$ (fact 103) \\
9902900716486180407546715254581773349090165822114492483005280554699876665841\\
6222832141441073883538492653516385977292093222882134415149891584000000000000\\
000000000000
    \end{tabular}
}

\subsection{説明}
\begin{enumerate}
    \item {\tt define}で{\tt (if 〜)}をグローバル変数{\tt fact}に定義し、またその引数として受け取った値を変数{\tt n}に格納します。
    \item 2行目以降の条件構文では{\tt n}が0以下ならば1を返し、それ以外なら{\tt n}と``{\tt fact}を{\tt n-1}を引数とした際の結果``の乗を返します。
    \item この部分は再帰的になっているので、$n=1$になるまで{\tt 2.}を繰り返します。\\
        つまり、{\tt (fact n)}は{\tt (fact n-1)}の結果に{\tt n}を掛けたものを結果とする。この時利用される{\tt (fact n-1)}の結果は{\tt n-1}に{\tt (fact n-2)}の結果を掛けたものになる。・・・というように繰り返し、{\tt (fact 1)}の結果として{\tt 1}と{\tt (fact 0)}の結果を必要とする処理に到達する。この時実行される{\tt (fact 0)}は再帰せずに{\tt 1}を結果として返す。\\
        ここから逆に辿って行くと$n(n-1)(n-2) \cdots 2\cdot1$という式と同じ計算が行われ、結果として階乗の値が導き出されます。
\end{enumerate}

\section{教科書1-1-4〜1-2-2の想定質問とその回答}

\subsection{1-1-4より}

\subsubsection{想定質問}

{\tt define}は合成手続きに名前を付けると説明されているが、名前は自由に命名することが可能であるのか?

\subsubsection{想定質問への回答}

Schemeには他のプログラミング言語でいうところの予約語({\em reserved keywords})は存在しません。\\
つまり、{\tt define}や{\tt if}などを{\tt define}で再定義することも許される。\\
例えば次のようなプログラムを作成実行することも可能です。\\
\\
{\bf プログラム例} \\
\fbox{
    \begin{tabular}{l}
(define \\
    (define x y)\\
    (* x y)\\
)
    \end{tabular}
}
\\
{\bf 実行例}
\\
\fbox{
    \begin{tabular}{l}
$>$(define 3 5)\\
15
\end{tabular}
}
\end{document}

