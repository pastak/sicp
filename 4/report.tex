\documentclass[a4paper,12pt]{article}
\usepackage{listings}
\begin{document}
\title{「アルゴリズムとデータ構造入門」\\
第4回課題}
\author{工学部情報学科\\
平成25年入学\\
学籍番号:1029-25-2723\\
森井 崇斗 }
\date{\today}
\maketitle

\lstset{numbers=left,basicstyle=\ttfamily\small,
  commentstyle=\textit, keywordstyle=\bfseries}

\section{ Fibonacci数の再帰型と繰り返し型手続きについて,ファイルを作成せよ}
fib.scm内に記述。\\
\begin{itemize}
    \item 再帰型:{\tt fib}
    \item 繰返型:{\tt fib-i}
\end{itemize}
\lstinputlisting{fib.scm}

\section{2種類のFibonacci数の手続きを実行し,fib(10),fib(20),fib(30) の出力結果を求めよ}

\fbox{
    \begin{tabular}{l}
$>$ (fib 10)\\
55\\
$>$ (fib 20)\\
6765\\
$>$ (fib 30)\\
832040\\
$>$ (fib-i 10)\\
55\\
$>$ (fib-i 20)\\
6765\\
$>$ (fib-i 30)\\
832040\\
    \end{tabular}
}

\section{ 2種類のFibonacci数の手続きを使ったfib(n) 実行時に fib が呼ばれる回数を,それぞれ解析的に求めよ.}

\subsection{fibが呼ばれる回数について}

{\tt fib}を呼び出した時、$n = 0 or 1$ の時は1回の呼び出しで終了する。\\
$n > 1$の時、{\tt fib(n)}実行時に{\tt fib}が呼ばれる回数を{\tt fibnum(n)}とすると、$ fibnum(n) = fibnum(n-1) + fibnum(n-2)$と表される。\\

\subsection{fib-i内のiterが呼ばれる回数について}

{\tt fib-i}を呼び出した時、内部で定義された手続きである{\tt iter}は{\tt fib-i(n)}に対して、{\tt n+1}回である。


\section{アッカーマン関数に関する練習問題}

\subsection{(ack 0 2)}
\begin{itemize}
    \item 2
\end{itemize}
\subsection{(ack 1 2)}
\begin{itemize}
    \item (ack 0 (ack 1 1))
    \item (ack 1 1) +1
    \item (ack 0 (ack 1 0)) +1
    \item (ack 1 0) +1 +1
    \item (ack 0 1) +2
    \item 2 +2
    \item 4
\end{itemize}
\subsection{(ack 2 2)}
\begin{itemize}
    \item (ack 1 (ack 2 1))
    \item (ack 1 (ack 1 (ack 2 0)))
    \item (ack 1 (ack 1 (ack 1 1)))
    \item (ack 1 (ack 1 (ack 0 (ack 1 0))))
    \item (ack 1 (ack 1 (ack 0 (ack 0 1))))
    \item (ack 1 (ack 1 (ack 0 2)))
    \item (ack 1 (ack 1 3))
    \item (ack 1 (ack 0 (ack 1 2)))
    \item (ack 1 (ack 0 4)) \#授業資料より(ack 1 2)の結果は4と導出済
    \item (ack 1 5)
    \item (ack 0 (ack 1 5))
    \item (ack 0 (ack 0 (ack 1 4)))
    \item (ack 0 (ack 0 (ack 0 (ack 1 3))))
    \item (ack 0 (ack 0 (ack 0 (ack 0 (ack 1 2)))))
    \item (ack 0 (ack 0 (ack 0 (ack 0 4))))
    \item (ack 0 (ack 0 (ack 0 5)))
    \item (ack 0 (ack 0 6))
    \item (ack 0 7)
    \item 8
\end{itemize}
\subsection{(ack 3 2)}
\begin{itemize}
    \item (ack 2 (ack 3 1))
    \item (ack 2 (ack 2 (ack 3 0)))
    \item (ack 2 (ack 2 (ack 2 1)))
    \item (ack 2 (ack 2 (ack 1 (ack 2 0))))
    \item (ack 2 (ack 2 (ack 1 (ack 1 1))))
    \item (ack 2 (ack 2 (ack 1 (ack 0 (ack 1 0)))))
    \item (ack 2 (ack 2 (ack 1 (ack 0 (ack 0 1)))))
    \item (ack 2 (ack 2 (ack 1 (ack 0 2))))
    \item (ack 2 (ack 2 (ack 1 3)))
    \item (ack 2 (ack 2 (ack 0 (ack 1 2))))
    \item (ack 2 (ack 2 (ack 0 4)))
    \item (ack 2 (ack 2 5))
    \item (ack 2 (ack 1 (ack 2 4)))
    \item (ack 2 (ack 1 (ack 1 (ack 2 3))))
    \item (ack 2 (ack 1 (ack 1 (ack 1 (ack 2 2)))))
    \item (ack 2 (ack 1 (ack 1 (ack 1 (ack 1 (ack 2 1))))))
    \item (ack 2 (ack 1 (ack 1 (ack 1 (ack 1 (ack 1 (ack 2 0)))))))
    \item (ack 2 (ack 1 (ack 1 (ack 1 (ack 1 (ack 1 (ack 1 1)))))))
    \item (ack 2 (ack 1 (ack 1 (ack 1 (ack 1 (ack 1 (ack 0 (ack 1 0))))))))
    \item (ack 2 (ack 1 (ack 1 (ack 1 (ack 1 (ack 1 (ack 0 (ack 0, 1))))))))
    \item (ack 2 (ack 1 (ack 1 (ack 1 (ack 1 (ack 1 (ack 0 2)))))))
    \item (ack 2 (ack 1 (ack 1 (ack 1 (ack 1 (ack 1 3))))))
    \item (ack 2 (ack 1 (ack 1 (ack 1 (ack 1 5)))))
    \item (ack 2 (ack 1 (ack 1 (ack 1 8))))
    \item (ack 2 (ack 1 (ack 1 (ack 0 (ack 1 7)))))
    \item (ack 2 (ack 1 (ack 1 (ack 0 (ack 0 (ack 1 6))))))
    \item (ack 2 (ack 1 (ack 1 (ack 0 (ack 0 (ack 0 (ack 1 5))))))
    \item (ack 2 (ack 1 (ack 1 (ack 0 (ack 0 (ack 0 8))))))
    \item (ack 2 (ack 1 (ack 1 (ack 0 (ack 0 9)))))
    \item (ack 2 (ack 1 (ack 1 (ack 0 10))))
    \item (ack 2 (ack 1 (ack 1 11)))
    \item (ack 2 (ack 1 13)) \# ここから中略
    \item (ack 2 15)
    \item 33
\end{itemize}
※ただし、これについては(ack 3 2)は29を返すべきであるので、途中で計算を誤っている。

\section{教科書 1-3-2 ~ 1-3-4 を読み、想定質問とそれへの解答と説明を記述せよ}

\subsection{想定質問}

教科書の該当部分ではlambdaを用いた手続きの構築について記述されていた。\\
lambdaと同等の仕組みとして、例えばJavascriptでは無名関数( {\it anonymous function} )という機能が実装されており、これによって、Schemeのlambda式を再現することが可能である。\\
例えば、教科書内で紹介されていた\\
$ f(x,y) = x(1+xy)^2 + y(1-y) + (1+xy)(1-y)$
lambda式を用いて実装するSchemeのコードをJavascriptの無名関数を用いて実装するとどのように書くことが出来るか。

\subsection{想定質問への解答と解説}
\subsubsection{想定質問への解答}
上記の計算式はJavascriptで無名関数を利用すると次のように記述できる。
\lstinputlisting{lambda.js}
\subsubsection{解説}

Javascriptでは無名関数という機能を提供している。\\
これは Scheme で書くと、\\
\fbox{
    (lambda ({\it $<$formal-parameters$>$} {\it $<$body$>$})
}\\
と同等のものを\\
\fbox{
    (function ({\it $<$formal-parameters$>$}) \{{\it $<$body$>$}\})
}\\
と記述することで再現可能にするものである。\\
また、Javascriptでこの記法を用いることで再現できることには、変数スコープが静的であることに由来している。\\
これを利用し、上記のように記述することでSchemeでlambdaを用いて実装したものをJavascriptで再実装することが可能である。


\end{document}

